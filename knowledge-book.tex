\documentclass[]{book}
\usepackage{lmodern}
\usepackage{amssymb,amsmath}
\usepackage{ifxetex,ifluatex}
\usepackage{fixltx2e} % provides \textsubscript
\ifnum 0\ifxetex 1\fi\ifluatex 1\fi=0 % if pdftex
  \usepackage[T1]{fontenc}
  \usepackage[utf8]{inputenc}
\else % if luatex or xelatex
  \ifxetex
    \usepackage{mathspec}
  \else
    \usepackage{fontspec}
  \fi
  \defaultfontfeatures{Ligatures=TeX,Scale=MatchLowercase}
\fi
% use upquote if available, for straight quotes in verbatim environments
\IfFileExists{upquote.sty}{\usepackage{upquote}}{}
% use microtype if available
\IfFileExists{microtype.sty}{%
\usepackage[]{microtype}
\UseMicrotypeSet[protrusion]{basicmath} % disable protrusion for tt fonts
}{}
\PassOptionsToPackage{hyphens}{url} % url is loaded by hyperref
\usepackage[unicode=true]{hyperref}
\hypersetup{
            pdftitle={Knowledge Book},
            pdfauthor={Sonia García-Ruiz (s.ruiz@ucl.ac.uk)},
            pdfborder={0 0 0},
            breaklinks=true}
\urlstyle{same}  % don't use monospace font for urls
\usepackage{natbib}
\bibliographystyle{apalike}
\usepackage{color}
\usepackage{fancyvrb}
\newcommand{\VerbBar}{|}
\newcommand{\VERB}{\Verb[commandchars=\\\{\}]}
\DefineVerbatimEnvironment{Highlighting}{Verbatim}{commandchars=\\\{\}}
% Add ',fontsize=\small' for more characters per line
\usepackage{framed}
\definecolor{shadecolor}{RGB}{248,248,248}
\newenvironment{Shaded}{\begin{snugshade}}{\end{snugshade}}
\newcommand{\KeywordTok}[1]{\textcolor[rgb]{0.13,0.29,0.53}{\textbf{#1}}}
\newcommand{\DataTypeTok}[1]{\textcolor[rgb]{0.13,0.29,0.53}{#1}}
\newcommand{\DecValTok}[1]{\textcolor[rgb]{0.00,0.00,0.81}{#1}}
\newcommand{\BaseNTok}[1]{\textcolor[rgb]{0.00,0.00,0.81}{#1}}
\newcommand{\FloatTok}[1]{\textcolor[rgb]{0.00,0.00,0.81}{#1}}
\newcommand{\ConstantTok}[1]{\textcolor[rgb]{0.00,0.00,0.00}{#1}}
\newcommand{\CharTok}[1]{\textcolor[rgb]{0.31,0.60,0.02}{#1}}
\newcommand{\SpecialCharTok}[1]{\textcolor[rgb]{0.00,0.00,0.00}{#1}}
\newcommand{\StringTok}[1]{\textcolor[rgb]{0.31,0.60,0.02}{#1}}
\newcommand{\VerbatimStringTok}[1]{\textcolor[rgb]{0.31,0.60,0.02}{#1}}
\newcommand{\SpecialStringTok}[1]{\textcolor[rgb]{0.31,0.60,0.02}{#1}}
\newcommand{\ImportTok}[1]{#1}
\newcommand{\CommentTok}[1]{\textcolor[rgb]{0.56,0.35,0.01}{\textit{#1}}}
\newcommand{\DocumentationTok}[1]{\textcolor[rgb]{0.56,0.35,0.01}{\textbf{\textit{#1}}}}
\newcommand{\AnnotationTok}[1]{\textcolor[rgb]{0.56,0.35,0.01}{\textbf{\textit{#1}}}}
\newcommand{\CommentVarTok}[1]{\textcolor[rgb]{0.56,0.35,0.01}{\textbf{\textit{#1}}}}
\newcommand{\OtherTok}[1]{\textcolor[rgb]{0.56,0.35,0.01}{#1}}
\newcommand{\FunctionTok}[1]{\textcolor[rgb]{0.00,0.00,0.00}{#1}}
\newcommand{\VariableTok}[1]{\textcolor[rgb]{0.00,0.00,0.00}{#1}}
\newcommand{\ControlFlowTok}[1]{\textcolor[rgb]{0.13,0.29,0.53}{\textbf{#1}}}
\newcommand{\OperatorTok}[1]{\textcolor[rgb]{0.81,0.36,0.00}{\textbf{#1}}}
\newcommand{\BuiltInTok}[1]{#1}
\newcommand{\ExtensionTok}[1]{#1}
\newcommand{\PreprocessorTok}[1]{\textcolor[rgb]{0.56,0.35,0.01}{\textit{#1}}}
\newcommand{\AttributeTok}[1]{\textcolor[rgb]{0.77,0.63,0.00}{#1}}
\newcommand{\RegionMarkerTok}[1]{#1}
\newcommand{\InformationTok}[1]{\textcolor[rgb]{0.56,0.35,0.01}{\textbf{\textit{#1}}}}
\newcommand{\WarningTok}[1]{\textcolor[rgb]{0.56,0.35,0.01}{\textbf{\textit{#1}}}}
\newcommand{\AlertTok}[1]{\textcolor[rgb]{0.94,0.16,0.16}{#1}}
\newcommand{\ErrorTok}[1]{\textcolor[rgb]{0.64,0.00,0.00}{\textbf{#1}}}
\newcommand{\NormalTok}[1]{#1}
\usepackage{longtable,booktabs}
% Fix footnotes in tables (requires footnote package)
\IfFileExists{footnote.sty}{\usepackage{footnote}\makesavenoteenv{long table}}{}
\usepackage{graphicx,grffile}
\makeatletter
\def\maxwidth{\ifdim\Gin@nat@width>\linewidth\linewidth\else\Gin@nat@width\fi}
\def\maxheight{\ifdim\Gin@nat@height>\textheight\textheight\else\Gin@nat@height\fi}
\makeatother
% Scale images if necessary, so that they will not overflow the page
% margins by default, and it is still possible to overwrite the defaults
% using explicit options in \includegraphics[width, height, ...]{}
\setkeys{Gin}{width=\maxwidth,height=\maxheight,keepaspectratio}
\IfFileExists{parskip.sty}{%
\usepackage{parskip}
}{% else
\setlength{\parindent}{0pt}
\setlength{\parskip}{6pt plus 2pt minus 1pt}
}
\setlength{\emergencystretch}{3em}  % prevent overfull lines
\providecommand{\tightlist}{%
  \setlength{\itemsep}{0pt}\setlength{\parskip}{0pt}}
\setcounter{secnumdepth}{5}
% Redefines (sub)paragraphs to behave more like sections
\ifx\paragraph\undefined\else
\let\oldparagraph\paragraph
\renewcommand{\paragraph}[1]{\oldparagraph{#1}\mbox{}}
\fi
\ifx\subparagraph\undefined\else
\let\oldsubparagraph\subparagraph
\renewcommand{\subparagraph}[1]{\oldsubparagraph{#1}\mbox{}}
\fi

% set default figure placement to htbp
\makeatletter
\def\fps@figure{htbp}
\makeatother

\usepackage{booktabs}

\title{Knowledge Book}
\author{Sonia García-Ruiz
(\href{mailto:s.ruiz@ucl.ac.uk}{\nolinkurl{s.ruiz@ucl.ac.uk}})}
\date{2021-08-17}

\begin{document}
\maketitle

{
\setcounter{tocdepth}{1}
\tableofcontents
}
\chapter{Analysis of Genetic Association
Studies}\label{analysis-of-genetic-association-studies}

The purpose of this section is to provide guidance on performing a GWAS
analysis step-by step, as well as introducing to useful software for
undertaking a GWAS.

\section{Useful Linux commands}\label{useful-linux-commands}

First, I'll show you useful Linux commands. Linux is an operating system
very well suited to dealing with large datasets.

\begin{itemize}
\tightlist
\item
  Create a shell script: touch script.sh
\end{itemize}

\chapter{GitHub}\label{github}

\section{Merge}\label{merge}

To merge two branches presenting conflicts, it is possible to indicate
which changes are preferred by using a --ours/--theirs flag:

First, we checkout to the branch that we want to apply/commit the merge:

\begin{verbatim}
> git checkout testing
> git log --oneline
\end{verbatim}

Merging would generate the following error:

\begin{verbatim}
> git merge testchanges2
warning: Cannot merge binary files: obj/Debug/netcoreapp2.1/CoExp_Web.csprojAssemblyReference.cache (HEAD vs. testchanges2)
warning: Cannot merge binary files: obj/Debug/netcoreapp2.1/CoExp_Web.Views.pdb (HEAD vs. testchanges2)
warning: Cannot merge binary files: obj/Debug/netcoreapp2.1/CoExp_Web.Views.dll (HEAD vs. testchanges2)
warning: Cannot merge binary files: .vs/CoExp_Web/v16/.suo (HEAD vs. testchanges2)
Auto-merging obj/Debug/netcoreapp2.1/CoExp_Web.csprojAssemblyReference.cache
CONFLICT (content): Merge conflict in obj/Debug/netcoreapp2.1/CoExp_Web.csprojAssemblyReference.cache
Auto-merging obj/Debug/netcoreapp2.1/CoExp_Web.Views.pdb
CONFLICT (content): Merge conflict in obj/Debug/netcoreapp2.1/CoExp_Web.Views.pdb
Auto-merging obj/Debug/netcoreapp2.1/CoExp_Web.Views.dll
CONFLICT (content): Merge conflict in obj/Debug/netcoreapp2.1/CoExp_Web.Views.dll
Auto-merging libman.json
CONFLICT (content): Merge conflict in libman.json
Auto-merging Views/Shared/_Layout.cshtml
CONFLICT (content): Merge conflict in Views/Shared/_Layout.cshtml
Auto-merging Views/Run/Help_Introduction.cshtml
Auto-merging Views/Run/Help_Catalogue.cshtml
Auto-merging Views/Run/Help_Annotation.cshtml
Auto-merging Views/Run/Help.cshtml
Auto-merging Views/Run/About.cshtml
Auto-merging .vs/CoExp_Web/v16/.suo
CONFLICT (content): Merge conflict in .vs/CoExp_Web/v16/.suo
Automatic merge failed; fix conflicts and then commit the result.
\end{verbatim}

To obtain the detail of the files that present a conflict:

\begin{verbatim}
> git log --merge
> git log --merge
commit 29cd34cc267a858fe696e86981e83bd3af1abb85 (testchanges2)
Author: Sonia Garcia <43370296+SoniaRuiz@users.noreply.github.com>
Date:   Tue Nov 24 09:03:14 2020 +0000

    commiting .suo file in testchanges2

commit 7319379b1038dc5415cf035f915d91096c70af2f (origin/testchanges2)
Author: Sonia Garcia <43370296+SoniaRuiz@users.noreply.github.com>
Date:   Tue Nov 24 00:47:04 2020 +0000

    remove unnecessary libraries

commit 5bf600b6f5f1b41b42faaca9173f33efac4958f4
Author: Sonia Garcia <43370296+SoniaRuiz@users.noreply.github.com>
Date:   Mon Nov 23 23:49:54 2020 +0000

    new built

commit 1f60b3ffb5b558849fa061efd7af1796380de30c
Author: Sonia Garcia <43370296+SoniaRuiz@users.noreply.github.com>
Date:   Mon Nov 23 23:20:51 2020 +0000

    Update .suo

    commit suo

commit d1834c7ed4a83a95e08a148c7945525e3521981c
Author: Sonia Garcia <43370296+SoniaRuiz@users.noreply.github.com>
Date:   Mon Nov 23 23:10:18 2020 +0000

> git merge --no-ff testchanges2
error: Merging is not possible because you have unmerged files.
hint: Fix them up in the work tree, and then use 'git add/rm <file>'
hint: as appropriate to mark resolution and make a commit.
fatal: Exiting because of an unresolved conflict.
\end{verbatim}

As we want to apply the comming changes, we use the flag ``--theirs'',
and the paths of the 7 conflictive files are updated:

\begin{verbatim}
> git checkout --theirs ./*
Updated 7 paths from the index
\end{verbatim}

Then, we commit the changes of those 7 files and push them to the
current repository (the `testing' repository):

\begin{verbatim}
> git branch
  master
  testchanges
  testchanges2
* testing
> git add ./*
> git commit -m "checkout --theirs testing"
[testing bcc3b5c] checkout --theirs testing
> git push
...
> git status
On branch testing
Your branch is up to date with 'origin/testing'.

nothing to commit, working tree clean
\end{verbatim}

Now, it's time to checkout to the repository we want to merge from and
apply the merge:

\begin{verbatim}
> git checkout testchanges2
Switched to branch 'testchanges2'
Your branch is ahead of 'origin/testchanges2' by 1 commit.
  (use "git push" to publish your local commits)

> git merge testing
Updating 29cd34c..bcc3b5c
Fast-forward
 Program.cs                         |  2 +-
 Properties/launchSettings.json     |  4 ++--
 Views/Run/About.cshtml             | 18 +++++++++---------
 Views/Run/Help.cshtml              |  4 ++--
 Views/Run/Help_Annotation.cshtml   |  4 ++--
 Views/Run/Help_Catalogue.cshtml    |  4 ++--
 Views/Run/Help_Introduction.cshtml |  4 ++--
 7 files changed, 20 insertions(+), 20 deletions(-)

> git checkout testing
Switched to branch 'testing'
Your branch is up to date with 'origin/testing'.

> git merge testchanges2
Already up to date.
\end{verbatim}

Once the merge is done, we can apply the commit to the `testing'
repository and push the changes. After that, both repositories
(`testchanges2' and `testing') could be considered as merged, being the
final result of their merge stored within the `testing' repository:

\begin{verbatim}
> git add ./*

> git commit -m "testing version with the plot"
[testing 7c2f7dd] testing version with the plot
 18 files changed, 90 deletions(-)
 ...

> git push origin testing
 ...
\end{verbatim}

\subsection{Troubleshooting pages of
interest}\label{troubleshooting-pages-of-interest}

\begin{itemize}
\tightlist
\item
  \href{https://stackoverflow.com/questions/13885390/in-git-merge-conflicts-how-do-i-keep-the-version-that-is-being-merged-in}{This}
\item
  \href{https://www.toolsqa.com/git/merge-branch-in-git/}{Merge a branch
  in Git}
\item
  \href{https://stackoverflow.com/questions/25576415/what-is-the-precise-meaning-of-ours-and-theirs-in-git\#:~:text=The\%20'ours'\%20in\%20Git\%20is,replayed\%20onto\%20the\%20current\%20branch}{Meaning
  of `ours' and `their' in Git}.)
\end{itemize}

\section{Stash}\label{stash}

When we are working on our GitHub project, we may want to backup some of
our changes. However, it is possible that those changes are not stable
yet, or we think they are not ready to be committed into our repository.
The stash command creates a backup copy of our changes, and keeps it
ready to be committed into the repo once we feel it is ready.

It might happen that, once we think the stash is ready to be committed,
GitHub doesn't allow us to do so because there are some `untracked'
files on it.

The only way I have found to deal with this issue, is to upload my stash
into a brand new repo, and finally merge the new repo with the one I was
intending to commit the changes to originally.

\chapter{Docker}\label{docker}

\section{Introduction}\label{introduction}

Nowadays, almost all companies and scientific labs which work with data
are also becoming software producers. In some cases, the main reason for
this software development is to publish effective scientific work. In
other cases, its final aim might be just putting together a set of
related functions that work for a similar final objective. However, in
both cases, the final software product becomes highly shareable.

This characteristic, the shareability of the software produced, might
seem very straightforward but in reality, it can turn into a scaring
daunting task to be solved. Why? Because of the heterogeneity of all
different platforms, operating systems, dependencies, versioning and so
on that our software product makes use of. Please, be aware that prior
to its first release, our software product might have been tested in a
limited number of PC stations with determined characteristics that can
be extremely different from any other potential user's PC station around
the world.

All these reasons make \href{https://www.docker.com/}{Docker Platform}
really interesting and useful for both software producers and consumers.

\subsection{What is docker?}\label{what-is-docker}

\href{https://www.docker.com/}{Docker} is a software platform created in
2013 by \href{https://en.wikipedia.org/wiki/Docker,_Inc.}{Docker, Inc.}
which its main objective is to \emph{build, share, and run any app
anywhere}, independently of the platform and environment where it is
executed.

But what does this definition mean? It means that you admiringly can
forget about dependencies, libraries, compatibility, etc. to make the
app run correctly. To some extent, you could think Docker as a
\emph{black box}, as a \textbf{snapshot} of the developer's laptop where
she or he developt the software you are about to make use of. A snapshot
of the precise moment when the developer decided to release the
software. Thus, everything you need to run the app is already there,
installed and configured inside the Docker object, ready to be used,
with no compatibility issues at all.

\subsection{Images and containers}\label{images-and-containers}

Working with Docker, there are two main concepts you are going to hear
about constantly: images and containers. An \textbf{image} is the
virtual file or template that contains the whole set of instructions to
build a container. An image is the raw Docker file you will directly
download from \href{https://hub.docker.com/}{Docker Hub} developer's
repository to your local. On the other hand, a \textbf{container} is the
executable object directly generated from the image. A container will be
the virtual object that represents the snapshot of the app developer's
laptop.

In summary, the image is the virtual file that contains the raw
instructions to build the executable app, and the executable app is the
container itself.

\subsection{Docker Hub}\label{docker-hub}

\href{https://hub.docker.com/}{Docker Hub} is an online platform that
allows creating individual Docker repositories. A Docker repository is a
personal account space where you can store one or more versions of the
same Docker image, which are represented by tags.

Let us focus on the following image obtained from the
\href{https://hub.docker.com/_/ubuntu?tab=tags}{Ubuntu Docker
Repository}:

\subsection{Useful Commands}\label{useful-commands}

This page contains a list with some of the most common commands of
Docker.

To download a Docker image from Docker Hub:

\begin{verbatim}
$ sudo docker push repository/name:tag
\end{verbatim}

To run the image \emph{name:tag}. The flag \textbf{--rm} indicates to
remove the container after stopping the image; whereas the flag
\textbf{-p} indicates the port on which we want to expose the execution
of the image:

\begin{verbatim}
$ sudo docker run --rm -p 8500:80 name:tag
\end{verbatim}

To list all docker images that are available in our local:

\begin{verbatim}
$ sudo docker images
\end{verbatim}

To remove the image \emph{image\_name}:

\begin{verbatim}
$ sudo docker image rm image_name
\end{verbatim}

To remove all orphaned images:

\begin{verbatim}
$ sudo docker rmi $(sudo docker images -f dangling=true -q)
\end{verbatim}

To list all current containers:

\begin{verbatim}
$ sudo docker ps
\end{verbatim}

To list all stopped containers:

\begin{verbatim}
$ sudo docker ps -a 
\end{verbatim}

To remove all orphaned containers:

\begin{verbatim}
$ sudo docker rm $(sudo docker ps -a -q)
\end{verbatim}

To enter inside a container in execution:

\begin{verbatim}
$ sudo docker exec -it name_container /bin/sh
\end{verbatim}

\section{CoExp}\label{coexp}

This tutorial contains the instructions to install a local version of
the CoExp Webpage by making use of the Docker technology. All the
commands shown in this document have been tested in a Ubuntu18.04
machine.

\subsection{Software requirements}\label{software-requirements}

Before downloading the CoExp Docker images, we first need to prepare the
environment for the correct execution of Docker.

Thus, let's download/fetch new versions of the packages that are already
installed in our Linux machine (all these commands have been tested in a
Ubuntu18.04 machine):

\begin{verbatim}
$ sudo apt update
$ sudo apt upgrade
\end{verbatim}

\emph{curl} is a tool used to transfer data. We will make use of it
later when we download the CoExp Docker images. To install \emph{curl}
in the machine:

\begin{verbatim}
$ sudo apt install curl
$ sudo curl --version
\end{verbatim}

Now, we are ready to install the Docker technology in the machine. So,
let's download it:

\begin{verbatim}
$ sudo apt install docker.io
\end{verbatim}

Once the Docker installation has finished, we can enable and start it.
The last instruction \emph{sudo docker --version} will return the
current Docker version installed:

\begin{verbatim}
$ sudo systemctl start docker
$ sudo systemctl enable docker
$ sudo docker --version
\end{verbatim}

Finally, we need to install Docker-compose (more info
\href{https://www.digitalocean.com/community/tutorials/how-to-install-docker-compose-on-ubuntu-18-04}{here}).
Docker-compose is a brach of the Docker technology, which allows
communicating different Docker images between them:

\begin{verbatim}
$ sudo curl -L https://github.com/docker/compose/releases/download/1.21.2/docker-compose-`uname -s`-`uname -m` -o /usr/local/bin/docker-compose
$ sudo chmod +x /usr/local/bin/docker-compose
$ docker-compose --version
\end{verbatim}

\subsection{Download Docker images of
CoExp}\label{download-docker-images-of-coexp}

If everything has gone as expected, the system should now be ready to
download the two CoExp Docker images. There are two different images
because one contains the user interface (UI) of the CoExp webpage, and
the other one contains the backend of the CoExp WebPage (author
\href{https://github.com/juanbot/CoExpNets}{Juan Botia}).

In terms of the back-end of the CoExp Webpage, there are two different
docker images available:

\begin{enumerate}
\def\labelenumi{\arabic{enumi}.}
\tightlist
\item
  Complete version: this docker image contains the totality of all CoExp
  networks, and it is about \textasciitilde{}4.5GB of size.
\item
  Lite version: this smaller docker image contains only the ROSMAP
  co-expression network. This image is about \textasciitilde{}1.3GB of
  size.
\end{enumerate}

Depending on which image you are interested in, the commands to execute
are:

To download the complete version:

\begin{verbatim}
$ sudo docker pull soniaruiz/coexp:r
\end{verbatim}

To download the lite version:

\begin{verbatim}
$ sudo docker pull soniaruiz/coexp:r-lite
\end{verbatim}

\subsection{Use Docker-Composer to build the
images}\label{use-docker-composer-to-build-the-images}

The next step is to make the comunication between the two docker images
possible. For that purpose, we need to download this
\href{https://github.com/SoniaRuiz/IPDGC/blob/master/complete/docker-compose.yml}{\textbf{docker-compose.yml}}
file (in case you have opted by the complete backend version), or this
\href{https://github.com/SoniaRuiz/IPDGC/blob/master/lite/docker-compose.yml}{\textbf{docker-compose.yml}}
file (if you have opted by the lite version). In any case, this file
will make possible the correct communication between the two Docker
images we downloaded in the previous step.

Additionally, the location of the downloaded \textbf{docker-compose.yml}
file is not really important, but we recommend to place it in your
\textbf{Home} folder (in case you are using a Linux machine).

Once the download has finished, use the following command to execute the
\textbf{docker-compose} file and, therefore, to run your own Docker
CoExp webpage:

\begin{verbatim}
$ sudo docker-compose up
\end{verbatim}

Finally, to test whether the execution has been correct, please type the
following URL into the address bar:

\begin{verbatim}
http://localhost:8088/
\end{verbatim}

If everything has gone as expected, you should now be able to visualize
your dockerized version of the CoExp webpage in your browser.
Congratulations!

\subsection{Juan's tutorial}\label{juans-tutorial}

Suppose we want to generate local TOM (Topology Overlap Measure) modules
from a specific network so we may, independently from the CoExpNets Web
application or within the application, plot or use networks in graph
mode. We can do it by creating their module TOMs and get the
corresponding graph in terms of a connectivity matrix we can plot.

We will exemplify this by using the frontal cortex network from GTEx V6
package as follows.

We launch all package stuff so we can start working with those networks.

\begin{Shaded}
\begin{Highlighting}[]
\KeywordTok{library}\NormalTok{(CoExpNets)}
\KeywordTok{library}\NormalTok{(CoExpGTEx)}
\NormalTok{CoExpGTEx}\OperatorTok{::}\KeywordTok{initDb}\NormalTok{()}

\NormalTok{netf =}\StringTok{ }\KeywordTok{getNetworkFromTissue}\NormalTok{(}\DataTypeTok{tissue=}\StringTok{"FCortex"}\NormalTok{,}
                            \DataTypeTok{which.on=}\StringTok{"gtexv6"}\NormalTok{,}
                            \DataTypeTok{only.file=}\NormalTok{T)}
\end{Highlighting}
\end{Shaded}

And now (we assume it is not generated yet) create the module-TOMs for
the Frontal Cortex Network as follows. As we see,

\begin{Shaded}
\begin{Highlighting}[]
\NormalTok{netf}
\end{Highlighting}
\end{Shaded}

the beta value for that network is 9, it is in the name between the
tissue and the \texttt{.it.50.rds} token in the file name.

\begin{Shaded}
\begin{Highlighting}[]
\KeywordTok{getModuleTOMs}\NormalTok{(}\DataTypeTok{tissue=}\StringTok{"FCortex"}\NormalTok{,}
              \DataTypeTok{beta=}\DecValTok{9}\NormalTok{,}
              \DataTypeTok{out.path=}\StringTok{"~/tmp/mytoms/"}\NormalTok{,}
              \DataTypeTok{which.one=}\StringTok{"gtexv6"}\NormalTok{)}
\end{Highlighting}
\end{Shaded}

And we can see all module-TOMs created now at the
\texttt{\textasciitilde{}/tmp/mytoms/} folder

\begin{Shaded}
\begin{Highlighting}[]
\KeywordTok{list.files}\NormalTok{(}\StringTok{"~/tmp/mytoms/"}\NormalTok{,}\DataTypeTok{full.names =}\NormalTok{ F,}\DataTypeTok{recursive =}\NormalTok{ F)}
\end{Highlighting}
\end{Shaded}

And now we can get any module's connectivity matrix so we can represent
a graph for the TOM

\begin{Shaded}
\begin{Highlighting}[]
\KeywordTok{getModuleTOMGraph}\NormalTok{(}\DataTypeTok{tissue=}\StringTok{"FCortex"}\NormalTok{,}
                  \DataTypeTok{which.one=}\StringTok{"gtexv6"}\NormalTok{,}
                  \DataTypeTok{module=}\StringTok{"black"}\NormalTok{,}
                  \DataTypeTok{topgenes=}\DecValTok{10}\NormalTok{,}
                  \DataTypeTok{out.path=}\StringTok{"~/tmp/mytoms/"}\NormalTok{)}
\end{Highlighting}
\end{Shaded}

And there you have it.

\chapter{Machine Learning Concepts}\label{machine-learning-concepts}

\section{Introduction}\label{introduction-1}

Have you ever asked yourself what is the difference between
\textbf{Artificial Intelligence} and \textbf{Machine Learning}? What
about between \textbf{supervised} and \textbf{unsupervised learning}?
Well, that's not surprising at all because trying to find out the right
answer within the huge ocean of information that is the Internet, can
become a really daunting task.

In this post, we will try first to define the most widely used Machine
Learning concepts and finally trying to give clarifying examples of each
one of them to try to help in the understanding of their meanings.

\section{Artificial Intelligence vs.~Machine
Learning}\label{artificial-intelligence-vs.machine-learning}

What is the difference between Artificial Intelligence and Machine
Learning? \textbf{Artificial Intelligence} is the concept of machines
being able to perform tasks in a way that we would consider ``smart''.
\textbf{Machine Learning} however is the current application of AI,
where we just give machines access to data and let them learn for
themselves (source
\href{https://www.forbes.com/sites/bernardmarr/2016/12/06/what-is-the-difference-between-artificial-intelligence-and-machine-learning/}{Forbes}).

The Machine Learning concept comprises different techniques whereby it
is possible to make machines learning from diverse sets of data. Among
the most important ones, we can highlight \textbf{supervised learning}
and \textbf{unsupervised learning}.

\section{Supervised learning}\label{supervised-learning}

When the training data - the data we want machines learning about -
comprises not only the input vectors but also their corresponding target
vectors.

\subsection{Classification}\label{classification}

Classification is a supervised learning method used when the target
vectors consist of a finite number of discrete categories.

The \emph{iris} dataset available in R, for instance, can be used in
classification problems because it provides different input vectors
(\emph{``Sepal.Length'', ``Sepal.Width'', ``Petal.Length'' and
``Petal.With''}) and a target vector (\emph{``Species''}) with a finite
number of categories (\emph{``setosa'', ``versicolor'' and
``virginica''}).

\begin{Shaded}
\begin{Highlighting}[]
\OperatorTok{>}\StringTok{ }\KeywordTok{summary}\NormalTok{(iris)}
\NormalTok{  Sepal.Length    Sepal.Width     Petal.Length    Petal.Width          Species  }
\NormalTok{ Min.   }\OperatorTok{:}\FloatTok{4.300}\NormalTok{   Min.   }\OperatorTok{:}\FloatTok{2.000}\NormalTok{   Min.   }\OperatorTok{:}\FloatTok{1.000}\NormalTok{   Min.   }\OperatorTok{:}\FloatTok{0.100}\NormalTok{   setosa    }\OperatorTok{:}\DecValTok{50}  
\NormalTok{ 1st Qu.}\OperatorTok{:}\FloatTok{5.100}\NormalTok{   1st Qu.}\OperatorTok{:}\FloatTok{2.800}\NormalTok{   1st Qu.}\OperatorTok{:}\FloatTok{1.600}\NormalTok{   1st Qu.}\OperatorTok{:}\FloatTok{0.300}\NormalTok{   versicolor}\OperatorTok{:}\DecValTok{50}  
\NormalTok{ Median }\OperatorTok{:}\FloatTok{5.800}\NormalTok{   Median }\OperatorTok{:}\FloatTok{3.000}\NormalTok{   Median }\OperatorTok{:}\FloatTok{4.350}\NormalTok{   Median }\OperatorTok{:}\FloatTok{1.300}\NormalTok{   virginica }\OperatorTok{:}\DecValTok{50}  
\NormalTok{ Mean   }\OperatorTok{:}\FloatTok{5.843}\NormalTok{   Mean   }\OperatorTok{:}\FloatTok{3.057}\NormalTok{   Mean   }\OperatorTok{:}\FloatTok{3.758}\NormalTok{   Mean   }\OperatorTok{:}\FloatTok{1.199}                  
\NormalTok{ 3rd Qu.}\OperatorTok{:}\FloatTok{6.400}\NormalTok{   3rd Qu.}\OperatorTok{:}\FloatTok{3.300}\NormalTok{   3rd Qu.}\OperatorTok{:}\FloatTok{5.100}\NormalTok{   3rd Qu.}\OperatorTok{:}\FloatTok{1.800}                  
\NormalTok{ Max.   }\OperatorTok{:}\FloatTok{7.900}\NormalTok{   Max.   }\OperatorTok{:}\FloatTok{4.400}\NormalTok{   Max.   }\OperatorTok{:}\FloatTok{6.900}\NormalTok{   Max.   }\OperatorTok{:}\FloatTok{2.500}  
\end{Highlighting}
\end{Shaded}

\subsection{Regression}\label{regression}

Regression is also a supervised learning method but only used when the
target vectors consist of one or more continuous variables.

The \emph{longley} R dataset is an example of this type of data. It
presents a collection of inputs vectors (\emph{``GNP.deflator'',
``GNP'', ``Unemployed'', ``Armed.Forces'', ``Population'', ``Year''})
and a numeric vector output (\emph{``Employed''}).

\begin{Shaded}
\begin{Highlighting}[]
\OperatorTok{>}\StringTok{ }\KeywordTok{summary}\NormalTok{(longley)}
\NormalTok{  GNP.deflator         GNP          Unemployed     Armed.Forces     Population         Year         Employed    }
\NormalTok{ Min.   }\OperatorTok{:}\StringTok{ }\FloatTok{83.00}\NormalTok{   Min.   }\OperatorTok{:}\FloatTok{234.3}\NormalTok{   Min.   }\OperatorTok{:}\FloatTok{187.0}\NormalTok{   Min.   }\OperatorTok{:}\FloatTok{145.6}\NormalTok{   Min.   }\OperatorTok{:}\FloatTok{107.6}\NormalTok{   Min.   }\OperatorTok{:}\DecValTok{1947}\NormalTok{   Min.   }\OperatorTok{:}\FloatTok{60.17}  
\NormalTok{ 1st Qu.}\OperatorTok{:}\StringTok{ }\FloatTok{94.53}\NormalTok{   1st Qu.}\OperatorTok{:}\FloatTok{317.9}\NormalTok{   1st Qu.}\OperatorTok{:}\FloatTok{234.8}\NormalTok{   1st Qu.}\OperatorTok{:}\FloatTok{229.8}\NormalTok{   1st Qu.}\OperatorTok{:}\FloatTok{111.8}\NormalTok{   1st Qu.}\OperatorTok{:}\DecValTok{1951}\NormalTok{   1st Qu.}\OperatorTok{:}\FloatTok{62.71}  
\NormalTok{ Median }\OperatorTok{:}\FloatTok{100.60}\NormalTok{   Median }\OperatorTok{:}\FloatTok{381.4}\NormalTok{   Median }\OperatorTok{:}\FloatTok{314.4}\NormalTok{   Median }\OperatorTok{:}\FloatTok{271.8}\NormalTok{   Median }\OperatorTok{:}\FloatTok{116.8}\NormalTok{   Median }\OperatorTok{:}\DecValTok{1954}\NormalTok{   Median }\OperatorTok{:}\FloatTok{65.50}  
\NormalTok{ Mean   }\OperatorTok{:}\FloatTok{101.68}\NormalTok{   Mean   }\OperatorTok{:}\FloatTok{387.7}\NormalTok{   Mean   }\OperatorTok{:}\FloatTok{319.3}\NormalTok{   Mean   }\OperatorTok{:}\FloatTok{260.7}\NormalTok{   Mean   }\OperatorTok{:}\FloatTok{117.4}\NormalTok{   Mean   }\OperatorTok{:}\DecValTok{1954}\NormalTok{   Mean   }\OperatorTok{:}\FloatTok{65.32}  
\NormalTok{ 3rd Qu.}\OperatorTok{:}\FloatTok{111.25}\NormalTok{   3rd Qu.}\OperatorTok{:}\FloatTok{454.1}\NormalTok{   3rd Qu.}\OperatorTok{:}\FloatTok{384.2}\NormalTok{   3rd Qu.}\OperatorTok{:}\FloatTok{306.1}\NormalTok{   3rd Qu.}\OperatorTok{:}\FloatTok{122.3}\NormalTok{   3rd Qu.}\OperatorTok{:}\DecValTok{1958}\NormalTok{   3rd Qu.}\OperatorTok{:}\FloatTok{68.29}  
\NormalTok{ Max.   }\OperatorTok{:}\FloatTok{116.90}\NormalTok{   Max.   }\OperatorTok{:}\FloatTok{554.9}\NormalTok{   Max.   }\OperatorTok{:}\FloatTok{480.6}\NormalTok{   Max.   }\OperatorTok{:}\FloatTok{359.4}\NormalTok{   Max.   }\OperatorTok{:}\FloatTok{130.1}\NormalTok{   Max.   }\OperatorTok{:}\DecValTok{1962}\NormalTok{   Max.   }\OperatorTok{:}\FloatTok{70.55}  
\end{Highlighting}
\end{Shaded}

\section{Unsupervised learning}\label{unsupervised-learning}

When the training data consists of a set of input vectors x without any
corresponding target values. The main aim of unsupervised learning
problems can be summarised in the following ideas.

\subsection{Clustering}\label{clustering}

Grouping the input data that share different similarities into clusters
- called \textbf{clustering} (a widely known clustering algorithm is
\href{https://stanford.edu/~cpiech/cs221/handouts/kmeans.html}{K-Means}).
Below, an example dataset which can be used in a clustering problem:

\begin{Shaded}
\begin{Highlighting}[]
\OperatorTok{>}\StringTok{ }\KeywordTok{library}\NormalTok{(cluster.datasets)}
\OperatorTok{>}\StringTok{ }\KeywordTok{data}\NormalTok{(acidosis.patients)}
\OperatorTok{>}\StringTok{ }\KeywordTok{summary}\NormalTok{(acidosis.patients)}
\NormalTok{ ph.cerebrospinal.fluid    ph.blood     hco3.cerebrospinal.fluid   hco3.blood    co2.cerebrospinal.fluid   co2.blood    }
\NormalTok{ Min.   }\OperatorTok{:}\FloatTok{38.50}\NormalTok{          Min.   }\OperatorTok{:}\FloatTok{32.80}\NormalTok{   Min.   }\OperatorTok{:}\StringTok{ }\FloatTok{9.80}\NormalTok{            Min.   }\OperatorTok{:}\StringTok{ }\FloatTok{4.20}\NormalTok{   Min.   }\OperatorTok{:}\FloatTok{17.80}\NormalTok{           Min.   }\OperatorTok{:}\FloatTok{12.90}  
\NormalTok{ 1st Qu.}\OperatorTok{:}\FloatTok{45.67}\NormalTok{          1st Qu.}\OperatorTok{:}\FloatTok{37.20}\NormalTok{   1st Qu.}\OperatorTok{:}\FloatTok{19.43}\NormalTok{            1st Qu.}\OperatorTok{:}\FloatTok{19.00}\NormalTok{   1st Qu.}\OperatorTok{:}\FloatTok{40.38}\NormalTok{           1st Qu.}\OperatorTok{:}\FloatTok{30.25}  
\NormalTok{ Median }\OperatorTok{:}\FloatTok{48.05}\NormalTok{          Median }\OperatorTok{:}\FloatTok{39.55}\NormalTok{   Median }\OperatorTok{:}\FloatTok{22.65}\NormalTok{            Median }\OperatorTok{:}\FloatTok{23.30}\NormalTok{   Median }\OperatorTok{:}\FloatTok{45.25}\NormalTok{           Median }\OperatorTok{:}\FloatTok{35.70}  
\NormalTok{ Mean   }\OperatorTok{:}\FloatTok{47.52}\NormalTok{          Mean   }\OperatorTok{:}\FloatTok{41.55}\NormalTok{   Mean   }\OperatorTok{:}\FloatTok{21.91}\NormalTok{            Mean   }\OperatorTok{:}\FloatTok{22.72}\NormalTok{   Mean   }\OperatorTok{:}\FloatTok{45.57}\NormalTok{           Mean   }\OperatorTok{:}\FloatTok{36.44}  
\NormalTok{ 3rd Qu.}\OperatorTok{:}\FloatTok{49.45}\NormalTok{          3rd Qu.}\OperatorTok{:}\FloatTok{41.98}\NormalTok{   3rd Qu.}\OperatorTok{:}\FloatTok{24.52}\NormalTok{            3rd Qu.}\OperatorTok{:}\FloatTok{26.93}\NormalTok{   3rd Qu.}\OperatorTok{:}\FloatTok{52.58}\NormalTok{           3rd Qu.}\OperatorTok{:}\FloatTok{42.30}  
\NormalTok{ Max.   }\OperatorTok{:}\FloatTok{54.90}\NormalTok{          Max.   }\OperatorTok{:}\FloatTok{81.30}\NormalTok{   Max.   }\OperatorTok{:}\FloatTok{30.40}\NormalTok{            Max.   }\OperatorTok{:}\FloatTok{34.80}\NormalTok{   Max.   }\OperatorTok{:}\FloatTok{69.00}\NormalTok{           Max.   }\OperatorTok{:}\FloatTok{61.40}  
\end{Highlighting}
\end{Shaded}

\includegraphics{https://datawookie.netlify.com/img/2015/09/xclara-clusters-colour.png}
Image source:
\href{https://datawookie.netlify.com/blog/2015/10/monthofjulia-day-30-clustering/}{MonthOfJulia
Day 30: Clustering}

\subsection{Density estimation}\label{density-estimation}

Determine the distribution of data within the input space, known as
\textbf{density estimation}. In other words, the aim of density
estimation is to use statistical models to find an underlying
probability distribution that is the main reason of the observed
variables. A widely known density estimation algorithm is Kernel Density
Estimation (KDE). Although KDE algorithm has a very intimidating name,
it can be used to visualize the continuous ``shape'' of some data
instead of using its discrete version through a histogram.

\includegraphics{https://blogs.sas.com/content/iml/files/2016/07/kdecomponents2.png}
Image source:
\href{https://blogs.sas.com/content/iml/2016/07/27/visualize-kernel-density-estimate.html}{How
to visualize a kernel density estimate}

\subsection{Visualization}\label{visualization}

This unsupervised method consist of projecting the data from a
high-dimensional space down to two or three dimensions for the purpose
of \textbf{visualization}.

\chapter{AWS}\label{aws}

\section{AWS S3}\label{aws-s3}

\subsection{Accessing AWS using the AWS Command Line
Interface}\label{accessing-aws-using-the-aws-command-line-interface}

The AWS Command Line Interface (or AWS CLI) is an open source tool that
enables you to interact with the AWS services from your command-line
Shell. AWS CLI tool is already installed in our server, and it is very
simple to use. Below are described the most commonly used interactions
with AWS.

\subsubsection{Configure the AWS CLI
service}\label{configure-the-aws-cli-service}

\begin{enumerate}
\def\labelenumi{\arabic{enumi}.}
\tightlist
\item
  To configure the service, please type ``aws configure'' on the
  console. This command is interactive, so the service will prompt you
  four times to enter some config information. Below an example (all
  RytenLab members have to ask me to send them their secret AWS
  credentials):
\end{enumerate}

\begin{verbatim}
$ aws configure 

AWS Access Key ID [None]: your_access_key 
AWS Secret Access Key [None]: your_secret_key 
Default region name [None]: eu-west-2 
Default output format [None]: json 
\end{verbatim}

\begin{enumerate}
\def\labelenumi{\arabic{enumi}.}
\setcounter{enumi}{1}
\tightlist
\item
  To check whether the connection with AWS has been correctly
  stablished, we can type the following command to list all our current
  buckets.
\end{enumerate}

\begin{verbatim}
$ aws s3 ls 
\end{verbatim}

\subsubsection{Upload a single file to
AWS}\label{upload-a-single-file-to-aws}

Let's suppose we have a bucket on AWS called `my-bucket'. Let's also
suppose you have a file called `myfile.txt' stored in your local that
you would like to upload to AWS. To upload the file `myfile.txt' to the
bucket `my-bucket':

\begin{verbatim}
$ aws s3 cp myfile.txt s3://my-bucket/ 
\end{verbatim}

\subsubsection{Download a single file from
AWS}\label{download-a-single-file-from-aws}

To download the file `myfile.txt' from the `s3://my-bucket/' AWS bucket
into your local folder:

\begin{verbatim}
$ aws s3 cp s3://my-bucket/myfile.txt ./my_local_folder 
\end{verbatim}

\subsubsection{Upload multiple files to
AWS}\label{upload-multiple-files-to-aws}

To upload multiple files, we can use the sync command. The \textbf{sync}
command syncs objects under a specified local folder to files in a AWS
bucket by uploading them to AWS.

\begin{verbatim}
$ aws s3 sync my_local_folder/ s3://my-bucket/ 
\end{verbatim}

\subsubsection{Download multiple files from
AWS}\label{download-multiple-files-from-aws}

To download multiple files from an AWS bucket to your local folder, we
can also use the \textbf{sync} command by changing the order of the
parameters.

\textbf{\emph{Please, be aware the costs associated with downloading
files correspond to \$0.090 per GB - first 10 TB / month data transfer
out beyond the global free tier.}}

\begin{verbatim}
$ aws s3 sync s3://my-bucket/my_remote_folder/ ./my_local_folder 
\end{verbatim}

\subsection{Checking AWS file
integrity}\label{checking-aws-file-integrity}

Considering the example used in the previous section, let's check the
integrity of the local folder `./my\_local\_folder' matches with the
integrity of the remote AWS folder `s3://my-bucket/my\_local\_folder/'.

\begin{enumerate}
\def\labelenumi{\arabic{enumi}.}
\tightlist
\item
  First, clone the `aws-s3-integrity-check' repo.
\end{enumerate}

\begin{verbatim}
$ git clone https://github.com/SoniaRuiz/aws-s3-integrity-check.git
\end{verbatim}

\begin{enumerate}
\def\labelenumi{\arabic{enumi}.}
\setcounter{enumi}{1}
\tightlist
\item
  Clone the `s3md5' repo.
\end{enumerate}

\begin{verbatim}
$ git clone https://github.com/antespi/s3md5.git
\end{verbatim}

\begin{enumerate}
\def\labelenumi{\arabic{enumi}.}
\setcounter{enumi}{2}
\tightlist
\item
  Move the s3md5 folder within the aws-s3-integrity-check folder:
\end{enumerate}

\begin{verbatim}
$ mv ./s3md5 ./aws-s3-integrity-check
\end{verbatim}

\begin{enumerate}
\def\labelenumi{\arabic{enumi}.}
\setcounter{enumi}{3}
\tightlist
\item
  Next, grant execution access permission to he s3md5 script file.
\end{enumerate}

\begin{verbatim}
$ chmod 755 ./aws-s3-integrity-check/s3md5/s3md5
\end{verbatim}

\begin{enumerate}
\def\labelenumi{\arabic{enumi}.}
\setcounter{enumi}{4}
\tightlist
\item
  The aws-s3-integrity-check folder should look similar to the
  following:
\end{enumerate}

\begin{verbatim}
total 16
-rw-r--r-- 1 your_user your_group 3573 date README.md
-rwxr-xr-x 1 your_user your_group 3301 date aws_check_integrity.sh
drwxr-xr-x 2 your_user your_group 4096 date s3md5
\end{verbatim}

\begin{enumerate}
\def\labelenumi{\arabic{enumi}.}
\setcounter{enumi}{5}
\tightlist
\item
  Execute the script `aws\_check\_integrity.sh' following the this
  structure: \textbf{aws\_check\_integrity.sh `local\_path'
  `bucket\_name' `bucket\_folder'}
\end{enumerate}

\begin{verbatim}
$ aws_check_integrity.sh ./my_local_folder my-bucket my_local_folder/
\end{verbatim}

\subsection{Creating a new AWS bucket}\label{creating-a-new-aws-bucket}

To create a new AWS bucket, we recommend using the following
configuration:

\begin{enumerate}
\def\labelenumi{\arabic{enumi}.}
\tightlist
\item
  Region: EU London
\item
  Block all public access: enabled
\item
  Bucket Versioning: enable
\item
  Tags:

  \begin{enumerate}
  \def\labelenumii{\arabic{enumii}.}
  \tightlist
  \item
    Key = ``data-owner'' / Value = ``your name''
  \item
    Key = ``data-origin'' / Value = ``the origin of the data in one word
    (i.e.~the name of a project, the name of a server)''
  \end{enumerate}
\item
  Default encryption: enable - Amazon S3 key (SSE-S3)
\item
  Advanced settings

  \begin{enumerate}
  \def\labelenumii{\arabic{enumii}.}
  \tightlist
  \item
    Object Lock: enable/disable, depending on your type of data (more
    info
    \href{https://docs.aws.amazon.com/AmazonS3/latest/dev/object-lock.html}{here})
  \end{enumerate}
\end{enumerate}

\section{AWS cloud services}\label{aws-cloud-services}

\begin{table}

\caption{\label{tab:unnamed-chunk-7}list of AWS cloud services.}
\centering
\begin{tabular}[t]{ll}
\toprule
name & description\\
\midrule
Amazon API Gateway & Service to develop APIs. These APIs will act as the front door for applications to access the data from the backend services behind the API.\\
Amazon Athena & SQL to analyse data stored in S3.\\
Amazon Aurora & MySQL-Compatible    Relational database for the cloud.\\
Amazon Aurora & PostgreSQL-Compatible   Relational database for the cloud.\\
Amazon Bracket & Quantum computing service.\\
\addlinespace
Amazon Carrier & A Carrier IP address is the address that I will assign to a network interface (for example an EC2 instance).\\
Amazon Chime & To let users meet and chat online.\\
Amazon CoudWatch & Monitoring and management service providing insights for AWS.\\
Amazon Code Guru Reviewer & Reviewer    Service that uses program analysis and machine learning to detect potential defects that are difficult for developers to detect.\\
Amazon Cognito & To add user sign-up, sign-in and access control to my web and mobile apps.\\
\addlinespace
Amazon Comprehend & For text processing and analysis.\\
Amazon Comprehend Medical & HIPAA-eligible natural language processing (NLP)\\
Amazon DocumentDB & Document database service\\
Amazon DynamoDB & Is a key-value and document database that delivers single-digit millisecond performance at any scale.\\
Amazon EC2 & Compute platform\\
\addlinespace
Amazon EC2 Dedicated Hosts & Allows you to use your eligible software licences on Amazon EC2.\\
Amazon EKS & Amazon Elastic Kubernetes Service\\
Amazon Elastic Block Store (EBS) & Allows you to create persistent block storage volumes and attach them to Amazon EC2 instances.\\
Amazon Elastic Container Registry (ECR) & To store, manage, and deploy Docker container images.\\
Amazon Elastic File System (EFS) & Provides a file system for use with AWS Cloud services.\\
\addlinespace
Amazon Elastic Graphics & Allows you to attach low-cost graphics acceleration to EC2 instances.\\
Amazon Elastic IP   Static & IPv4 for dynamic cloud computing.\\
Amazon Elastic Transcoder & Is media transcoding in the cloud. Designed to convert (or “transcode”) media files from their source format into versions that will playback on devices like smartphones.\\
Amazon ElastiCache & Build data-intensive apps or improve the performance of your existing apps by retrieving data from high throughput and low latency in-memory data stores.\\
Amazon Elasticsearch Service & To operate Elasticsearch at scale with zero downtime.\\
\addlinespace
Amazon EMR & Industry-leading cloud big data platform for processing vast amounts of data using open-source tools. EMR makes easy to set up, operate, and scale your big-data environments by automating time-consuming tasks like provisioning capacity and tuning clusters. To run petabyte-scale analysis at less than half of the cost of traditional on-premises solutions.\\
Amazon FSx for Lustre & Integrated with S3, is designed for fast processing of workloads.\\
Amazon FSx for Windows File Server & To move your Windows based applications that require file storage to AWS.\\
Amazon GuardDuty & Is a threat detection service that monitors for malicious activity to protect your AWS accounts and data stored in Amazon S3.\\
Amazon Kinesis Data Analytics & To analyse streaming data (for responding to your business and customer needs in real time)\\
\addlinespace
Amazon Kinesis Data Firehose & To only pay for the volume of data ingested into the service.\\
Amazon Kinesis Data streams & Data streaming service.\\
Amazon Lookout for Vision & Machine learning (ML) service that spots defects and anomalies in visual representations using computer vision (CV).\\
Amazon Managed Streaming for Apache Kafka (MSK) & To easily run applications that use Apache Kafka\\
Amazon MQ & To set up and operate message brokers in the cloud.\\
\addlinespace
Amazon Neptune & Graph database service to build and run applications that work with highly connected datasets.\\
Amazon Polly & Service that turns text into lifelike speech.\\
Amazon QuickSight & Business intelligence service to deliver insights to everyone in your organization.\\
Amazon RDS for MariaDB & To set up, operate and scale MariaDB database deployments in the cloud.\\
Amazon RDS for MySQL & ''\\
\addlinespace
Amazon RDS for Oracle & \\
Amazon RDS for PostgreSQL & \\
Amazon RDS for SQL server & \\
Amazon Redshift & Petabyte-scale data warehouse\\
Amazon Rekognition & Images and video analysis recognition; to identify objects, people, text, scenes, etc. Also, to detect inappropriate content.\\
\addlinespace
Amazon Route 53 & DNS web service\\
Amazon S3 Glacier & \\
Amazon SageMaker & To manage costs associated to ML instances\\
Amazon SageMaker Ground Truth & To build training datasets for machine learning.\\
Amazon Simple Email Service & Cloud-based email sending service.\\
\addlinespace
Amazon Simple Notification Service & Amazon Simple Notification Service\\
Amazon Simple Queue Service (SQS) & Amazon Simple Queue Service (SQS)\\
Amazon S3 & Amazon Simple Storage Service (S3).\\
Amazon Simple Workflow Service (SWF) & Amazon Simple Workflow Service (SWF)\\
\bottomrule
\end{tabular}
\end{table}

\bibliography{book.bib,packages.bib}

\end{document}
